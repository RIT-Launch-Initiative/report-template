\def\AbstractDescription
{
	At a minimum, the abstract shall identify the launch vehicles
	mission/category, identify any unique/defining characteristics, define the
	payload's mission (if applicable), and provide whatever additonal
	information may be necessary to convey any other high-level project or
	program goals \& objectives.
}

\def\IntroductionDescription
{
	This section provides an overview of the academic program, stakeholders,
	team structure, and team management strategies. The introduction may repeat
	some of the content included in the abstract, because the abstract is
	intended to act as a standalone synopsis if necessary.
}

\def\SAODescription
{
	This section shall begin with a top-level overview of the integrated system,
	including a cutaway figure depicting the fully integrated launch vehicle and
	its major subsystems--configured for the mission being flown in the
	competition. This description shall be followed by the follwing subsections.
	Each subsection shall include detailed descriptions of each subsystem, and
	reflect the technical analysis used to support design and manufacturing
	decisions. Technical drawings of these subsystems should be included in the
	specified appendix.
}

\def\CONOPSDescription
{
	This section shall identify the mission phases, include a figure, and
	describe the nominal operation of all subsystems during each phase (e.g. a
	description of what is supposed to be occuring in each phase, and what
	subsystem[s] are responsible for accomplishing this). Furthermore, this
	section shall define what mission events signify a phase transition has
	occured (e.g., ``Ignition'' may begine when a FIRE signal is sent to the
	igniter and conclude when the propulsion system comes up to chamber
	pressure. Similarly, ``Liftoff'' may begine at vehicle first motion and
	conclude when the vehicleis free of the launch rail). Phases and phase
	transitions are expected to vary from system to system based on specific
	design implementations and mission goals \& objectives. No matter how a team
	defines these mission phases, and phase transitions, they will be used to
	help organize failure modes identified in a Risk Assessment Appendix.
}

\def\ConclusionDescription
{
	This section shall include the lessons learned during the design,
	manufacture, and testing of the project, both from a team management and
	technical perspective. Furthermore, this section should include strategies
	for corporate knowledge transfer from senior student team members to the
	rising upperclassmen who will soon take their place. 
}

\def\SystemMeasuresDescription
{
	This requirement will be satisfied by appending the Third/Final Progress
	Report as the first appendix of the Project Technical Report. As described
	in Section 2.7.1 of this document, the Third/Final Progress Report is also
	submitted as a separate excel file for administrative purposes.
}

\def\TestReportsDescription
{
	The second Project Technical Report appendix shall contain applicable Test
	Reports from the minimum tests prescribed in the \href{IREC Design, Test,
	and Evaluation Guide}{http://www.soundingrocket.org/sa-cup-documents--forms.html}. 
	These reports shall appear in the following order. In the event any report
	is not applicable to the project in question, the team will include a page
	marked ``THIS PAGE INTENTIONALLY LEFT BLANK'' in its place.

	\begin{itemize}
		\item Recovery System Testing: In addition to descriptions of testing
			performed and the results thereof, teams shall include in this
			appendix a figure and supporting text describing the dual redundancy
			of recovery system electronics.
		\item SRAD Propulsion System Testing (if applicable): In addition to
			descriptions of testing performed and the results thereof, teams
			developing SRAD hybrid or liquid propulsion systems shall include in
			this appendix a fluid circuit diagram. This figure shall identify
			nominal operating pressures at various key points in the system –
			including the fill system.
		\item SRAD Pressure Vessel Testing (if applicable).
	\end{itemize}
}

\def\HazardAnalysisDescription
{
	This appendix shall address as applicable, hazardous material handling,
	transportation and storage procedures of propellants, and any other aspects
	of the design which pose potential hazards to operating personnel. A
	mitigation approach--by process and/or design--shall be defined for each
	hazard identified. An example of such a matrix is available on \href{the
	ESRA website}{http://www.soundingrocket.org/sa-cup-documents--forms.html}.
}

\def\RiskAssessmentDescription
{
	This appendix shall summarize risk and reliability concepts
	associated with the project. All identified failure modes which pose a risk
	to mission success shall be recorded in a matrix, organized according to the
	mission phases identified by the CONOPS. A mitigation approach--by process
	and/or design--shall be defined for each risk identified. An example of
	such a matrix is available on the\href{the ESRA
	website}{http://www.soundingrocket.org/sa-cup-documents--forms.html}.
}

\def\ChecklistsDescription
{
	This appendix shall include detailed checklist procedures for final
	assembly, arming, launch, and recovery operations. Furthermore, these
	checklists shall include alternate process flows for dis-arming/safeing the
	system based on identified failure modes. These off-nominal checklist
	procedures shall not conflict with the IREC Range Standard Operating
	Procedures. Teams developing SRAD hybrid or liquid propulsion systems shall
	also include in this appendix a description of processes and procedures used
	for cleaning all propellent tanks and other fluid circuit components.

	Competition officials will verify teams are following their checklists
	during all operations--including assembly, preflight, launch, and recovery
	operations.  Therefore, teams shall maintain a complete, hardcopy set of
	these checklist procedures with their flight hardware during all range
	activities.
}

\def\DrawingsDescription
{
	The sixth Project Technical Report appendix shall contain Engineering
	Drawings.  This appendix shall include any revision controlled technical
	drawings necessary to define significant subsystems or components--
	especially SRAD subsystems or components.
}

\def\AcknowledgementsDescription
{
	An Acknowledgments section, if used, \textbf{immediately precedes} the
	References. Sponsorship information and funding data are included here. The
	preferred spelling of the word ``acknowledgment'' in American English is
	without the ``e'' after the ``g.'' Avoid expressions such as ``One of us
	(S.B.A.) would like to thank\ldots'' Instead, write ``F.~A.~Author
	thanks\ldots'' Sponsor and financial support acknowledgments are also to be
	listed in the ``acknowledgments'' section.
}
